\documentclass[10pt]{article}
\usepackage{zsj}

\usepackage{silence}
\WarningFilter{latex}{Marginpar on page}

\allowdisplaybreaks

\hbadness=99999

\newcommand{\me}{\mathrm{e}}
%\newcommand{\rr}{\mathbb{R}}
\newcommand{\ii}{\mathrm{i}}
%\newcommand{\md}{\mathrm{d}}
%\DeclareMathOperator{\im}{im}

\usepackage{empheq}
\tcbset{
  highlight math style={
    enhanced,
    colframe=NavyBlue!70!black,
    colback=NavyBlue!10,
    boxrule=1pt,
  }
}
\newenvironment{boxedalign}
  {\empheq[box=\tcbhighmath]{align}}
  {\endempheq}

\usepackage{annotate-equations}


\begin{document}
\title{Conformal Field Theory}
\subheader{Notes}
\author{Shangjie Zhou\orcidlink{0000-0001-9576-5011}}
\affiliation{School of Physics and Technology, Wuhan University}
\emailAdd{sjzhou@whu.edu.cn}
\abstract{\textit{Last updated on: \today}\\Repository: \url{https://github.com/spaceofzsj/Notes}\\Personal Website: \url{https://spaceofzsj.github.io/ShangjieZhou/}\\Blog: \url{https://spaceofzsj.github.io/}}

\maketitle

\section*{Introduction}
\addcontentsline{toc}{section}{\protect\numberline{}Introduction}
A conformal field theory (CFT) is a quantum field theory that is invariant under \textit{conformal transformations}.

Conformal field theory has important applications to condensed matter physics, statistical mechanics, quantum statistical mechanics, and string theory. 

This notes are mainly based on \cite{DiFrancesco:1997nk}.

There are also useful lecture notes like \cite{Qualls:2015qjb,Tong:2009np}.

\part{Review}
\section{Quantum Field Theory}
\subsection{Symmetries and Conservation Laws}
\subsubsection{Continuous Symmetry Transformations}
The new action is 
\begin{align}
    S'&=\int\dd[d]{x}\mathcal{L}\left(\Phi'(x),\partial_\mu\Phi'(x)\right)\notag\\
        &=\int\dd[d]{x'}\mathcal{L}\left(\Phi'(x'),\partial'_\mu\Phi'(x')\right)\notag\\
        &=\int\dd[d]{x'}\mathcal{L}\left(\mathcal{F}\left(\Phi(x)\right),\partial'_\mu\mathcal{F}(\Phi(x))\right)\notag\\
        &=\int\dd[d]{x}\abs{\pdv{x'}{x}}\mathcal{L}\left(\mathcal{F}(\Phi(x)),\left(\pdv*{x^\nu}{x'^\mu}\right)\partial_\nu\mathcal{F}(\Phi(x))\right).\label{eq:CST:new_action}
\end{align}
\subsubsection{Infinitesimal Transformations and Noether's Theorem}
Infinitesimal transformations may in general written as 
\begin{align}
    x'^\mu&=x^\mu+\omega_a\fdv{x^\mu}{\omega_a}\label{eq:CST:x_prime}\\
    \Phi'(x')&=\Phi(x)+\omega_a \fdv{\mathcal{F}(x)}{\omega_a}\label{eq:CST:phi_prime} 
\end{align}
Here $\{\omega_a\}$ is a set of infinitesimal parameters, which will be kept to first order only.
\begin{definition}[Generator of a symmetry transformation]\label{def:generator_of_a_symmetry_transformation}
    The \textit{generator} $G_a$ of a symmetry transformation is defined by the following expression for the infinitesimal transformation at a same point:
    \begin{align}
        \delta_\omega\Phi(x)\equiv\Phi'(x)-\Phi(x)\equiv-\ii\omega_a G_a\Phi(x).
    \end{align}
\end{definition}
Notice that 
\begin{align}
    \Phi'(x')=\Phi(x)+\omega_a \fdv{\mathcal{F}(x)}{\omega_a}=\Phi(x')-\omega_a\fdv{x^\mu}{\omega_a}\partial_\mu\Phi(x')+\omega_a\fdv{\mathcal{F}(x')}{\omega_a}
\end{align}
The explicit expression of the generator is therefore 
\begin{align}
    \ii G_a\Phi=\fdv{x^\mu}{\omega_a}\partial_\mu\Phi-\fdv{\mathcal{F}}{\omega_a}
\end{align}
\begin{example}[Infinitesimal translation]
    For an infinitesimal translation by a vector $\omega_\mu$, one has $\fdv*{x^\mu}{\omega^\nu}=\delta^\mu_\nu$ and $\fdv*{\mathcal{F}}{\omega^\nu}=0$.
    The generator of translations is simply
    \begin{align}
        P_\nu=-\ii\partial_\nu.
    \end{align}
\end{example}
\begin{theorem}[Noether's theorem]
    Every continuous symmetry of the action is associated with a current that is \textit{classically}\snm conserved. 
\end{theorem}
\snt{Notice the \textit{classically}.}
\begin{proof}
    Using \cref{eq:CST:x_prime,eq:CST:phi_prime} in \cref{eq:CST:new_action}, we have 
    \begin{align}
        \pdv{x'^\mu}{x^\mu}=\delta^\nu_\mu+\partial_\mu\left(\omega_a\fdv{x^\nu}{\omega_a}\right).
    \end{align}
\end{proof}
\subsubsection{Transformations of the Correlation Functions}
\subsubsection{Ward Identities}
\subsection{The Energy-Momentum Tensor}
Consider the infinitesimal translation\sidenote{The field is unaffected by the translation.} $x^\mu\to x^\mu+\epsilon^\mu$, we have 
\begin{align}
    \fdv{x^\mu}{\epsilon^\mu}=\delta^\mu_\nu\quad\fdv{\mathcal{F}(x)}{\epsilon^\mu}=0.
\end{align}
\subsubsection{The Belinfante Tensor}
\section{Statistical Mechanics}
\subsection{The Boltzmann Distribution}
\subsection{Critical Phenomena}
\subsection{The Renormalization Group: Lattice Models}
\subsection{The Renormalization Group: Continuum Models}
\subsection{The Transfer Matrix}

\part{Fundamentals}
\section{Global Conformal Invariance}
\subsection{The Conformal Group}
We denote by $g_{\mu\nu}$ the metric tensor in a spacetime of dimension $d$.
\begin{definition}[Conformal transformation]
    A conformal transformation of the coordinates is an invertible mapping $x\to x'$, for which the induced metric tensor invariant up to a scale:
    \begin{align}
        g'_{\mu\nu}(x)=\Lambda(x)g_{\mu\nu}(x).
    \end{align}
\end{definition}
The set of conformal transformations manifestly forms a group, and it obviously has the Poincar\'{e} group as a subgroup, since the latter corresponds to the special case $\Lambda(x)=1$.
\subsubsection{Infinitesimal Conformal Transformation}
Consider an infinitesimal conformal transformation $x^\mu\to x'^{\mu}+\epsilon^\mu(x)$.
The induced metric is 
\begin{align}
    g'_{\mu\nu}=g_{\mu\nu}-\left(\partial_\mu\epsilon_\nu+\partial_\nu\epsilon_\mu\right).
\end{align}
The requirement that the transformation be conformal implies that 
\begin{align}
    \partial_\mu\epsilon_\nu+\partial_\nu\epsilon_\mu=f(x)g_{\mu\nu}.\label{eq:GCI:fx}
\end{align}
Taking the trace of both sides of \cref{eq:GCI:fx}, we have 
\begin{align}
    f(x)=\frac{2}{d}\partial_\rho\epsilon^\rho.\label{eq:GCI:fx1}
\end{align}

For simplicity, we assume the original metric is standard Cartesian metric $g_{\mu\nu}=\eta_{\mu\nu}$, where $\eta_{\mu\nu}=\mathrm{diag}(1,1,\dots,1)$.\sidenote{Of cource we can consider Minkowski metric. The treatment is identical, except for the explicit form of $\eta_{\mu\nu}$.}
By applying an extra derivative $\partial_\rho$ on \cref{eq:GCI:fx}, permuting the indices and taking a linear combination, we arrive at
\begin{align}
    2\partial_\mu\partial_\nu\epsilon_\rho=\eta_{\mu\rho}\partial_\nu f+\eta_{\nu\rho}\partial_\mu f-\eta_{\mu\nu}\partial_\rho f.\label{eq:GCI:fx2}
\end{align}
Upon contracting with $\eta^{\mu\nu}$, \cref{eq:GCI:fx2} becomes 
\begin{align}
    2\partial^2\epsilon_\mu=(2-d)\partial_\mu f \label{eq:GCI:fx3}
\end{align}
Applying $\partial_\nu$ on \cref{eq:GCI:fx3} and $\partial^2$ on \cref{eq:GCI:fx}, we have respectively
\begin{align}
    \partial^2\partial_\nu\epsilon_\mu=&\frac{2-d}{2}\partial_\mu\partial_\nu f\label{eq:GCI:fx4}\\
    \eta_{\mu\nu}\partial^2 f=&\partial^2\partial_\mu\epsilon_\nu+\partial^2\partial_\nu\epsilon_\mu.\label{eq:GCI:fx5}
\end{align} 
Combine \cref{eq:GCI:fx4,eq:GCI:fx5}, we have\sidenote{Notice that the $\mu$ and $\nu$ indices in \cref{eq:GCI:fx4} are symmetric.}
\begin{align}
    (2-d)\partial_\mu\partial_\nu f=\eta_{\mu\nu}\partial^2 f.\label{eq:GCI:fx6}
\end{align}
Finally, contracting \cref{eq:GCI:fx6} with $\eta^{\mu\nu}$, we end up with 
\begin{align}
    (d-1)\partial^2 f=0.\label{eq:GCI:fx7}
\end{align}
\paragraph{The case $d=1$}
The above equations do not impose any constraint on the function $f$, and therefore any smooth transformation is conformal in one dimension.
\paragraph{The case $d=2$}
This case is special and it will be studied in detail later.
\paragraph{The case $d\geq3$}
Now \cref{eq:GCI:fx6,eq:GCI:fx7} imply that $\partial_\mu\partial_\nu f=0$ and therefore 
\begin{align}
    f(x)=A+B_\mu x^\mu\label{eq:GCI:fxs}
\end{align}
where $A$ and $B_\mu$ are constants.
If we substitute \cref{eq:GCI:fxs} into \cref{eq:GCI:fx2}, we see that $\partial_\mu\partial_\nu\epsilon_\rho$ is constant, which means we have the general expression 
\begin{align}
    \epsilon_\mu=a_\mu+b_{\mu\nu}x^\nu+c_{\mu\nu\rho}x^\nu x^\rho\qq{where}c_{\mu\nu\rho}=c_{\mu\rho\nu}.
\end{align}
$a_\mu$, $b_{\mu\nu}$ and $c_{\mu\nu\rho}$ are constants.
We can treat each power of the coordinate separately and use the constraints \cref{eq:GCI:fx,eq:GCI:fx1,eq:GCI:fx2}:
\begin{itemize}
    \item The constant term $a_\mu$ is free of constraints.
    This term amounts to an infinitesimal translation.
    \item Substitution of the linear term into \cref{eq:GCI:fx,eq:GCI:fx1} yields 
          \begin{align}
            b_{\mu\nu}+b_{\nu\mu}=\frac{2}{d}\tensor{b}{^\lambda_\lambda}\eta_{\mu\nu}
          \end{align}
          which implies that\sidenote{We can always decompose a matrix into a sum of a symmetric part and an anti-symmetric part by $a_{mn}=\frac{a_{mn}+a_{nm}}{2}+\frac{a_{mn}-a_{nm}}{2}$.} 
          \begin{align}
            b_{\mu\nu}=\alpha\eta_{\mu\nu}+m_{\mu\nu}\qq{where}m_{\mu\nu}=-m_{\nu\mu}.
          \end{align}
          $\alpha$ is a constant.
          The symmetric part represents an infinitesimal scale transformation, whereas the anti-symmetric part is an infinitesimal rigid rotation.
    \item Substitution of the linear term into \cref{eq:GCI:fx2,eq:GCI:fx1} yields 
          \begin{align}
            c_{\mu\nu\rho}=\eta_{\mu\rho}b_{\nu}+\eta_{\mu\nu}b_{\rho}-\eta_{\nu\rho}b_{\mu}\qq{where}b_{\mu}\equiv\frac{\tensor{c}{^\sigma_\sigma_\mu}}{d}
          \end{align} 
          and the corresponding infinitesimal transformation is 
          \begin{align}
            x'^\mu=x^\mu+2(x\vdot b)x^\mu-b^\mu x^2
          \end{align}
          which is called \textit{special conformal transformation} (SCT).
\end{itemize}
The finite transformations corresponding to the above are the following:
\begin{boxedalign}
    \text{(translation)}&\quad x'^\mu=x^\mu+a^\mu\\
    \text{(dilation)}&\quad x'^\mu=\alpha x^\mu\\
    \text{(rigid rotation)}&\quad x'^\mu=\tensor{M}{^\mu_\nu}x^\nu\\
    \text{(SCT)}&\quad x'^\mu=\frac{x^\mu-b^\mu x^2}{1-2b\vdot x+b^2 x^2}
\end{boxedalign}

Recall the \cref{def:generator_of_a_symmetry_transformation} and suppose the field are unaffected by the transformation, we can get the generators of the conformal group 
\begin{boxedalign}
    \text{(translation)}&\quad P_\mu=-\ii\partial_\mu\\
    \text{(dilation)}&\quad D=-\ii x^\mu\partial_\mu\\
    \text{(rigid rotation)}&\quad L_{\mu\nu}=\ii(x_\mu\partial_\nu-x_\nu\partial_\mu)\\
    \text{(SCT)}&\quad K_\mu=-\ii(2x_\mu x^\nu\partial_\nu-x^2\partial_\mu).
\end{boxedalign}
The commutators of these generators define the \textit{conformal algebra}
\begin{boxedalign}
    \comm{D}{P_{\mu}}&=\ii P_\mu\label{eq:conformal_algebra_1}\\
    \comm{D}{K_\mu}&=-\ii K_\mu\label{eq:conformal_algebra_2}\\
    \comm{K_\mu}{P_\nu}&=2\ii\left(\eta_{\mu\nu}D-L_{\mu\nu}\right)\label{eq:conformal_algebra_3}\\
    \comm{K_\rho}{L_{\mu\nu}}&=\ii(\eta_{\rho\mu}K_\nu-\eta_{\rho\nu}K_\mu)\label{eq:conformal_algebra_4}\\
    \comm{P_\rho}{L_{\mu\nu}}&=\ii(\eta_{\rho\mu}P_{\nu}-\eta_{\rho\nu}P_{\mu})\label{eq:conformal_algebra_5}\\
    \comm{L_{\mu\nu}}{L_{\rho\sigma}}&=\ii\left(\eta_{\nu\rho}L_{\mu\sigma}+\eta_{\mu\sigma}L_{\nu\rho}-\eta_{\mu\rho}L_{\nu\sigma}-\eta_{\nu\sigma}L_{\mu\rho}\right).\label{eq:conformal_algebra_6}
\end{boxedalign}
Other commutators vanish.
\begin{remark}
    The number of the generators of the conformal group is
    \begin{align}
        &1\ \mathrm{dilation}+d\ \mathrm{translations}+d\ \mathrm{SCTs}+\frac{d(d-1)}{2}\ \mathrm{rotations}\notag\\
        &=\frac{(d+2)(d+1)}{2}\ \mathrm{generators},
    \end{align}
    which is exactly the number of generators of the group $\mathrm{SO}(d+1,1)$.

    In order to put the above commutation rules into a simpler form, we define the following generators\snm: 
    \begin{align}
        J_{\mu,\nu}&\equiv L_{\mu\nu}\\
        J_{-1,\mu}&\equiv \frac{1}{2}(P_\mu-K_\mu)\\
        J_{0,\mu}&\equiv \frac{1}{2}(P_\mu+K_\mu)\\
        J_{-1,0}&\equiv D\\
        J_{a,b}&\equiv-J_{b,a},
    \end{align}
    where $a,b\in\{-1,0,1,\dots,d\}$.
    The last definition implies the diagonal elements vanish.
    These new generators obey the $\mathrm{SO}(d+1,1)$ algebra 
    \begin{align}
        \comm{J_{a,b}}{J_{c,d}}=\ii\left(\eta_{ad}J_{b,c}+\eta_{bc}J_{a,d}-\eta_{ac}J_{b,d}-\eta_{bc}J_{a,c}\right)
    \end{align}
    where the diagonal metric $\eta_{ab}$ is $\mathrm{diag}(-1,1,1,\dots,1)$ if spacetime is Euclidean (otherwise an additional component, say $g_{dd}$ is negative).

    This shows the isomorphism between the $d$-dimensional conformal group and $\mathrm{SO}(d+1,1)$. 
\end{remark}
\snt{Notice that the $\mu$ and $\nu$ indices runs from 1 to $d$.}



\subsection{Conformal Invariance in Classical Field Theory}
\subsubsection{Representations of the Conformal Group in \texorpdfstring{$d$}{d} Dimensions}
\begin{intu}
Given an infinitesimal conformal transformation parametrized by $\omega_g$, we seek a matrix representation $T_g$ that a multicomponent field $\Phi(x)$ transforms as 
\begin{align}
    \Phi'(x')=(1-\ii\omega_g T_g)\Phi(x).
\end{align}
In the previous parts, we only considered the generators of the conformal transformations that only include spacetime transformation and leave the field unaffected.
Now we are considering conformal transformations that include field transformations, which is called the \textit{full conformal transformation}.
The corresponding generators are called the \textit{full generators} of the symmetry. 
\end{intu}
We first look at an example of Poincar\'e transformation
\begin{example}
    We define the effect of Lorentz transformation generator $L_{\mu\nu}$ on the field at $x=0$ by

    \begin{align}
        L_{\mu\nu}\Phi(0)\equiv \eqnmarkbox[blue]{node1}{S_{\mu\nu}} \Phi(0).
    \end{align}
    \annotate[yshift=0.5em]{}{node1}{Matrix}
    $S_{\mu\nu}$ is an ordinary matrix since Lorentz transformation maps the origin to itself.
    We can use a trick to get the effect of $L_{\mu\nu}$ on fields elsewhere
    \begin{align}
        L_{\mu\nu}\Phi(x)&=L_{\mu\nu}\me^{\ii x^\mu P_\mu}\Phi(0)\notag\\
                        &=\me^{\ii x^\mu P_\mu}\me^{-\ii x^\mu P_\mu}L_{\mu\nu}\me^{\ii x^\mu P_\mu}\Phi(0)\label{eq:poincare_algebra_trick_haussdorff}\\
                        &=\me^{\ii x^\mu P_\mu}\left(L_{\mu\nu}-x_\mu P_{\nu}+x_\nu P_\mu\right)\Phi(0)\notag\\
                        &=\me^{\ii x^\mu P_\mu}\left(S_{\mu\nu}-x_\mu P_{\nu}+x_\nu P_\mu\right)\Phi(0)\notag\\
                        &=\left(S_{\mu\nu}-x_\mu P_{\nu}+x_\nu P_\mu\right)\Phi(x).
    \end{align}
    We used the \textit{Hausdorff formula} in \cref{eq:poincare_algebra_trick_haussdorff}
\begin{align}\label{eq:haussdorff}
    \me^{-A}B\me^{A}=B+\comm{B}{A}+\frac{1}{2!}\comm{\comm{B}{A}}{A}+\frac{1}{3!}\comm{\comm{\comm{B}{A}}{A}}{A}+\dots.
\end{align}
We can do this for $P_\mu$ in the same way. 
And finally 
\begin{align}
    P_\mu\Phi(x)&=-\ii\partial_\mu\Phi(x)\\
    L_{\mu\nu}\Phi(x)&=\ii(x_\mu \partial_\nu-x_\nu\partial_\mu)\Phi(x)+S_{\mu\nu}\Phi(x).    
\end{align}
\end{example}
\begin{remark}
    If we already know how a generator act on field at one point, we can use the translation operators to derive the its effect on fields at other points.
    In particular, if we know the effect of an operator $\mathcal{O}$ on the field at origin $\Phi(0)$, then its effect on $\Phi(x)$ can be calculated from $\me^{\ii x^\mu P_\mu}\mathcal{O}\me^{-\ii x^\mu P_\mu}$.
\end{remark}
We can in the same way for the full conformal group.
We denote the effect of $L_{\mu\nu}$, $D$ and $K_\mu$ at $x=0$ by $S_{\mu\nu}$, $\tilde{\Delta}$ and $\kappa_\mu$\sidenote{$S_{\mu\nu}$, $\tilde{\Delta}$ and $\kappa_\mu$ are all matrices since rotations, dilations and SCTs all map the origin to itself.}.
They form a matrix representation of the reduced conformal algebra as in \cref{eq:conformal_algebra_2,eq:conformal_algebra_4,eq:conformal_algebra_6}
\begin{align}
    \comm{\tilde{\Delta}}{S_{\mu\nu}}=&0\\
    \comm{\tilde{\Delta}}{\kappa_\mu}=&-\ii\kappa_\mu\\
    \comm{\kappa_\nu}{\kappa_\mu}=&0\\
    \comm{\kappa_\rho}{S_{\mu\nu}}=&\ii\left(\eta_{\rho\mu}\kappa_{\nu}-\eta_{\rho\nu}\kappa_\mu\right)\\
    \comm{S_{\mu\nu}}{S_{\rho\sigma}}=&\ii\left(\eta_{\nu\rho}S_{\mu\sigma}+\eta_{\mu\sigma}S_{\nu\rho}-\eta_{\mu\rho}S_{\nu\sigma}-\eta_{\nu\sigma}S_{\mu\rho}\right).
\end{align}
Using \cref{eq:conformal_algebra_1,eq:conformal_algebra_2,eq:conformal_algebra_3,eq:conformal_algebra_4,eq:conformal_algebra_5,eq:conformal_algebra_6,eq:haussdorff}, we have 
\begin{align}
    \me^{\ii x^\rho P_\rho}D\me^{-\ii x^\rho P_\rho}=&D+x^\nu P_\nu\\
    \me^{\ii x^\rho P_\rho}K_\mu\me^{-\ii x^\rho P_\rho}=&K_\mu+2x_\mu D-2x^\nu L_{\mu\nu}+2x_\mu\left(x^\nu P_\nu\right)-x^2 P_\mu
\end{align}
and therefore
\begin{align}
    D\Phi(x)=&\left(-\ii x^\nu\partial_\nu+\tilde{\Delta}\right)\Phi\\
    K_\mu\Phi(x)=&\left(\kappa_\mu+2x_\mu\tilde{\Delta}-x^\nu S_{\mu\nu}-2\ii x_\mu x^\nu\partial_\nu+\ii x^2 \partial_\mu\right)\Phi(x).
\end{align}

\subsubsection{The Energy-Momentum Tensor}
Under an arbitrary transformation of the coordinates $x^\mu\to x^\mu+\epsilon^\mu(x)$, the action changes as follows 
\begin{align}
    \delta S&=\int\dd[d]{x}T^{\mu\nu}\partial_\mu\epsilon_\nu \notag\\
            &=\frac{1}{2}\int\dd[d]{x}T^{\mu\nu}\left(\partial_\mu\epsilon_\nu+\partial_\nu\epsilon_\mu\right)
\end{align}
where the energy-momentum tensor $T^{\mu\nu}$ is assumed to be symmetric.

Under an infinitesimal conformal transformation, using \cref{eq:GCI:fx,eq:GCI:fx1}, we have 
\begin{align}
    \delta S&=\frac{1}{2}\int\dd[d]{x}T^{\mu\nu}f(x)g_{\mu\nu}\notag\\
            &=\frac{1}{d}\int\dd[d]{x}\tensor{T}{^\mu_\mu}\partial_\rho\epsilon^\rho.
\end{align}
\begin{intu}
    Under certain conditions, the energy-momentum tensor of a theory with scale invariance can be made traceless.
    If this is possible, then it follows from the above that full conformal invariance is a consequence of scale invariance and Poincar\'e invariance.
\end{intu}


\subsection{Conformal Invariance in Quantum Field Theory}
\subsubsection{Correlation Functions}
Consider the two-point function 
\begin{align}
    \expval{\phi_1(x_1)\phi_2(x_2)}=\frac{1}{Z}\int\left[\dd{\Phi}\right]\phi_1(x_1)\phi_2(x_2)\exp(-S\left[\Phi\right])
\end{align}
\section{Conformal Invariance in Two Dimensions}
\subsection{The Conformal Group in Two Dimensions}
\subsection{Ward Identities}
\subsection{Free Fields and the Operator Product Expansion}

\clearpage
\bibliographystyle{jhep}
\bibliography{ref}
\end{document}